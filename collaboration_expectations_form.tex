\setlength\extrarowheight{4pt}

\begin{table}[!ht]
   \caption{Collaboration expectations form. The idea here is that there are no good answers (well, sometimes there are), but that each party fills the form and then compares their answer, to hopefully align expectations. Some questions are phrased rhetorically, but meant to explicitly prepare all members for a collaboration's potential tension points.}
    \label{form}
    \raggedleft
    \begin{adjustbox}{max width=\textwidth}
    \begin{tabular}{|p{3in}|l|l|l|l|l|p{3in}|}
    \hline
        ~ & 1 & 2 & 3 & 4 & 5 & ~ \tabularnewline \hline
        \textbf{Choosing collaborators} & ~ & ~ & ~ & ~ & ~ & ~ \tabularnewline \hline
        The questions on this form must be adhered to as agreed & ~ & ~ & ~ & ~ & ~ & Questions on this form are only for orientation \\ \hline
        Scope and whether to continue collaborating should be discussed regularly & ~ & ~ & ~ & ~ & ~ & Once the collaboration has started, it will be seen through to a deliverable, no matter what happens \\ \hline
        \textbf{Experimental design} & ~ & ~ & ~ & ~ & ~ & ~ \\ \hline
        The wet lab people determine the experimental design/strategy, the dry lab people determine the analysis strategy & ~ & ~ & ~ & ~ & ~ & Both experimental and analysis strategies are to be discussed and jointly agreed on by both parties \\ \hline
        Analysis plan and experimental design are to be discussed once the data has been collected & ~ & ~ & ~ & ~ & ~ & Analysis plan and experimental design should be discussed and fixed prior to data collection \\ \hline
        Analysts are blinded whenever possible & ~ & ~ & ~ & ~ & ~ & All metadata is always visible to analysts \\ \hline
        \textbf{Communication and functioning} & ~ & ~ & ~ & ~ & ~ & ~ \\ \hline
        Meetings should be held on a regular (e.g. monthly/biweekly) basis, and simply be brief if there was no development & ~ & ~ & ~ & ~ & ~ & Meetings should be organized whenever they are needed (e.g. to discuss results or problems) \\ \hline
        Both parties should always adhere to agreed deadlines & ~ & ~ & ~ & ~ & ~ & Agreed deadlines are guidelines rather than absolute targets \\ \hline
        To each his/her expertise & ~ & ~ & ~ & ~ & ~ & All parties are expected to try to learn and understand what the others are doing \\ \hline
        As much as possible, every piece of data/analysis should be made available to everyone & ~ & ~ & ~ & ~ & ~ & Results are to be exchanged/communicated between partners in the form of publication-ready figures and descriptions \\ \hline
        Emails are expected to be answered on the same day & ~ & ~ & ~ & ~ & ~ & Emails are expected to be answered within a week or two \\ \hline
        Both parties agree on tools to track progress (i.e. Trello) and collaborative writing (i.e. Google Docs, Overleaf) & ~ & ~ & ~ & ~ & ~ & Each team uses their preferred software \\ \hline
        Predefined file naming conventions & ~ & ~ & ~ & ~ & ~ & No convention but file names are clear \\ \hline
        Metadata delivered in machine readable format & ~ & ~ & ~ & ~ & ~ & Metadata sent as is collected \\ \hline
        Manuscript written in a collaborative document allowing concurrent editing & ~ & ~ & ~ & ~ & ~ & Different versions of manuscripts are individual documents shared by email \\ \hline
        \textbf{Scientific values and publication strategy} & ~ & ~ & ~ & ~ & ~ & ~ \\ \hline
        What matters is excellent science - and hence high impact factor & ~ & ~ & ~ & ~ & ~ & What matters is sound, robust science \\ \hline
        We should make code/data available if we are forced to & ~ & ~ & ~ & ~ & ~ & By default everything should be openly available, latest by publication time (whether pre-print or final paper) \\ \hline
        Everything should be published as pre-print & ~ & ~ & ~ & ~ & ~ & Never mention anything unless it’s actually published by a journal \\ \hline
        The people doing the experiments are first authors, the dry lab people are somewhere after that & ~ & ~ & ~ & ~ & ~ & The people analyzing and making sense of the data are first authors or first co-authors \\ \hline
        Negative or inconclusive results are sadly unglamorous, but happen and are part of science & ~ & ~ & ~ & ~ & ~ & If you have negative or inconclusive results, it means you haven't looked hard enough or given it proper attention \\ \hline
        \textbf{Intellectual contribution and attribution} & ~ & ~ & ~ & ~ & ~ & ~ \\ \hline
        Projects are owned and conceived by the wet lab & ~ & ~ & ~ & ~ & ~ & Projects are owned and conceived by the dry lab \\ \hline
        Results are interpreted by the dry lab & ~ & ~ & ~ & ~ & ~ & Results are interpreted by the wet lab \\ \hline
        Figure making is a graphic design task (using an image editor) & ~ & ~ & ~ & ~ & ~ & Figure making is a data analysis task (using ggplot, GraphPad, etc) \\ \hline
        \textbf{Etiquette} & ~ & ~ & ~ & ~ & ~ & ~ \\ \hline
        Direct communication is not harsh, just direct & ~ & ~ & ~ & ~ & ~ & Well thought, polite phrasing is expected in all communication exchanges \\ \hline
        Excluding collaboration members from some exchanges (meetings, messages) is ok & ~ & ~ & ~ & ~ & ~ & All exchanges should be public and no member can be excluded \\ \hline
        Freedom to give feedback regardless of career stage & ~ & ~ & ~ & ~ & ~ & Mind status when giving feedback \\ \hline
    \end{tabular}

    \end{adjustbox}
%    \begin{tablenotes}
%      \small
%      \item Supplementary document: collaboration expectations form.
%    \end{tablenotes}
\end{table}