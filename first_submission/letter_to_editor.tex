\documentclass[a4paper]{article}

\usepackage[margin=1in]{geometry}

\usepackage[utf8]{inputenc}
\usepackage{hyperref}
\hypersetup{
	unicode,
}

\begin{document}

\noindent Dear Editor,\\

We would like to submit our draft ``\textit{Ten simple rules for collaborating with wet lab researchers for computational researchers}'' to PLoS Computational Biology.\\

In this Ten Simple Rules manuscript, we explore scientific collaborations with (biology) experimentalists from the perspective of a dry lab research group that is primarily
engaged in assessing and developing new computational tools. We are aware of other Ten Simple Rules papers discussing bioinformatics support \cite{kumuthini2020ten}, managing bioinformatics capacities \cite{aron2021ten} and wider cross-disciplinary collaborations (including service work) \cite{knapp2015ten}. As compared to those, our manuscript focuses on scientific collaborations among equals. To our understanding, our viewpoint is rather mainstream in scientific practice albeit underrepresented in educational resources such as Ten Simple Rules guidelines. We list common sources of misalignments and provide actionable items to ease these collaborations at both the scientific and operational levels.\\

We think our paper matches PLoS Computational Biology's call for educational resources addressing current problems in computational biology and the journal wide readership.\\

Thank you in advance for processing our submission. \\

Best regards,\\

Mark D. Robinson and Izaskun Mallona\\
\indent University of Zurich\\

\vspace{1cm}

\textit{Suggested reviewers}

\begin{itemize}
    \item Bernhard Knapp, \texttt{bernhard.knapp@technikum-wien.at}, corresponding author of `Ten Simple Rules for a Successful Cross-Disciplinary Collaboration' \cite{knapp2015ten} - a guideline on the human side of establishing and maintaining cross-disciplinary collaborations.
    \item Monika Cechova, \texttt{cechova.biomonika@gmail.com}, unique author of `Ten simple rules for biologists initiating a collaboration with computer scientists' \cite{cechova2020ten} which addresses the complementary challenge of aligning with computational experts from a wet lab perspective.
\end{itemize}

\vspace{1cm}

\renewcommand{\section}[2]{}%
\bibliographystyle{plain}  
\bibliography{letter_to_editor}  


\end{document}