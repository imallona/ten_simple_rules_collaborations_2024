\documentclass[a4paper]{article}

\usepackage[margin=1in]{geometry}

\usepackage[utf8]{inputenc}
\usepackage{hyperref}
\hypersetup{
	unicode,
}

\begin{document}

\noindent Dear Editor,\\

Thank you for your fast handling of our manuscript. Please find below a point to point response to reviewers' comments.


\begin{quote}
    Reviewer #1: My main concern is that this does not sufficiently differ from the previous 2015 paper “Ten Simple Rules for a Successful Cross-Disciplinary Collaboration”. Pretty much every provided idea is already covered in the previous paper.

\end{quote}

We agree the advice from the `Ten Simple Rules for a Successful Cross-Disciplinary Collaboration' paper is valuable and could be applied to some of intended readers, particularly if also engaged in support roles. Other overlapping papers exist.

We note that our aim is different from those papers. We address scientific projects where computational biologists and experimentalists collaborate to carry out biology projects; to our understanding, this is not interdisciplinary (anymore). Finally, we don't contemplate bioinformatics/IT service work (rule 7 of the cross-disciplinary paper), as we address life sciences projects carried out by equals (some of which are wet lab resarchers, and others dry lab researchers)

We have updated the introduction to clarify our scope and motivation, citing Knapp et al. as well as other complementary papers.

\begin{quote}
    Reviewer #2: Review of “10 Simple Rules for Computational Researchers Collaborating with Wet Lab Researchers”

I think this review is useful and I probably should have read something like it before engaging in my most recent collaboration where the goal posts of what kinds of data would be included in a visualization tool my lab was developing have annoyingly been shifting over the past year and a half. Mission creep is not fun.

I think the article is a bit long and at times jargony. The title could be more clearly phrased, for one thing (as above). And what is CNS (on Page 2; I know it is Cell-Nature-Science but I am not sure if this is a commonly understood acronym), for instance? What does “multi-silverback-group-leader” mean (Page 1)? I think a better way of describing competing egos is to say just that. Some of text could be tightened up.


\end{quote}

Thank you for the positive comments. We have replaced added an explanation to `CNS` and removed the silverbacks.

\begin{quote}
    The authors alternate between using “data” as a singular noun and as a plural noun (“How and what data is shared internally…” and in the next paragraph “Sensitive data…represent a special case”. Be consistent.

\end{quote}

Thank you, we have harmonized `data' as a plural noun.


\begin{quote}
    For Rule 2, it is not just Europe that has sensitive data regulations. For instance, PIPEDA is in force in Canada, and the US has HIPAA.

\end{quote}

Thank you very much, we have included those.

\begin{quote}
    Is the title for Rule 6 missing some commas? “Communicate early, openly, and often enough”

\end{quote}

Agreed, we have added commas.


\begin{quote}
    I don’t think the “analysis” in “analysis’ code” needs to be possessive in the actionable item of Rule 10.

\end{quote}

Agreed, we have updated "your analysis" to "the analysis".

\begin{quote}
    
Maybe a figure would be handy to summarize the 10 actionable items? It could be simply the 10 rule headings in boxes and e.g. Trello and GitHub logos beside Rule 5. This would make it easy for someone to show to collaborators at the start of a potential collaboration :-)

\end{quote}

Mark?

\begin{quote}
    
The table is a bit confusing. The scale is like a 5 point Likert scale but there are two statements. Maybe some instruction at the top to put an “x” in the appropriate column, where 3 is “agree equally with both statements”, 1 is “agree with statement on left strongly”, and 5 is “agree with statement on right strongly”.

\end{quote}

Agreed, we have updated the table header to ease filling it out.

\end{document}



Fully agree it's confusing, we can edit the table header with the reviewer's instructions.

Reviewer #3: This article is a well-written and well-thought piece, written by authors who clearly have lots of experience with interdisciplinary research collaboration. I am excited to see it published in the future. I especially enjoyed and appreciated Table 1, and might use it in my own research practice. The quotes also nicely illustrated presented points.

My only suggestion is to consider adding a few personal anecdotes from the past collaborations, that would illustrate the rules and what happens when they are followed and/or broken.

No way. What do you think?

Minor comments:
1)
"Without an aligned choice of collaborating labs, discrepancies in scientific interest and values are harder to navigate later on" => not clear what choice refers to here, please explain or elaborate

Sure.

2)
"What matters is excellent science - and hence high impact factor vs What matters is sound, robust science" => I think I understand the dichotomy the authors are presenting here, but I would suggest re-phrasing, as I can see two people with similar mindset choosing either option, especially in current publishing culture

We were being cynical (bad thing) but I'd rather keep the cynicism, open to discuss how.

3)
Is font size in Table 1 consistent?

The row sizes were meant to be flexible, but font size was meant to be fixed; I'll check.

4) "To each his/her expertise" => could this be rephrased to be more inclusive, e.g. "their expertise"?

Agree

5) Initially when reading the article, it was a bit confusing whether the advice was targeted from wet-lab to dry-lab scientists, or the other direction, and it felt like both directions were mentioned at times. Perhaps this could be unified.

We could update the title to clarify, the current one is not very elegant. Suggestions?

\end{document}