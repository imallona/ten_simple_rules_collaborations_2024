\documentclass[a4paper]{article}

\usepackage[margin=1in]{geometry}


\usepackage[utf8]{inputenc}
\usepackage{hyperref}
\hypersetup{
	unicode,
}

\usepackage{etoolbox}
\AtBeginEnvironment{quote}{\itshape}


\begin{document}

\noindent Dear Editor,\\

Thank you for the fast handling of our manuscript. Please find below a point-to-point response to the reviewers' comments, as well as the highlighted and clean and updated manuscript versions.\\

We have also added `actionable items' to rule 1, which we missed in the first submission.\\

Thank you in advance for processing our submission. \\

Best regards,\\

\noindent Izaskun Mallona and Mark D. Robinson \\
\noindent University of Zurich

\vspace{1cm}

\noindent \textbf{Reviewer 1}

\begin{quote}
    Reviewer \#1: My main concern is that this does not sufficiently differ from the previous 2015 paper `Ten Simple Rules for a Successful Cross-Disciplinary Collaboration'. Pretty much every provided idea is already covered in the previous paper.

\end{quote}

Thank you for your feedback. We agree the advice from `Ten Simple Rules for a Successful Cross-Disciplinary Collaboration' by Knapp et al is valuable and could be applied to some of the intended readers. We note other partly overlapping yet distinct `ten simple rules' papers exist.

We note that our aim is different from those. We aim to address scientific projects where computational biologists and experimentalists collaborate: to our understanding, this is a standard in biology nowadays and not interdisciplinary anymore. Similarly, we do not contemplate bioinformatics/IT service/support work (e.g. rule 7 of the cross-disciplinary paper by Knapp et al). Our intended readers are computational biologists involved in life science projects carried out among equals. 
In addition to the rules, we also do provide a tool in form of a questionnaire (table 1), to be used in practice for aligning expectations between the collaborating parties.

We have updated the introduction to clarify our scope and motivation, citing Knapp et al as well as other complementary papers giving more context and also providing a direct link to other resources orthogonal to ours, but potentially helpful to bioinformatics support readers or people starting a bioinformatics facility.\\

\noindent \textbf{Reviewer 2}

\begin{quote}
    Reviewer \#2: Review of `10 Simple Rules for Computational Researchers Collaborating with Wet Lab Researchers'

I think this review is useful and I probably should have read something like it before engaging in my most recent collaboration where the goal posts of what kinds of data would be included in a visualization tool my lab was developing have annoyingly been shifting over the past year and a half. Mission creep is not fun.

I think the article is a bit long and at times jargony. The title could be more clearly phrased, for one thing (as above). And what is CNS (on Page 2; I know it is Cell-Nature-Science but I am not sure if this is a commonly understood acronym), for instance? What does `multi-silverback-group-leader' mean (Page 1)? I think a better way of describing competing egos is to say just that. Some of text could be tightened up.


\end{quote}

Thank you for the positive comments and for the editing suggestion. We have replaced `CNS` and removed/rephrased the silverback comment.

\begin{quote}
    The authors alternate between using `data' as a singular noun and as a plural noun (`How and what data is shared internally…' and in the next paragraph `Sensitive data…represent a special case'. Be consistent.

\end{quote}

Agreed, we have harmonized `data' as a plural noun throughout the text.


\begin{quote}
    For Rule 2, it is not just Europe that has sensitive data regulations. For instance, PIPEDA is in force in Canada, and the US has HIPAA.

\end{quote}

Thank you, we have included those.

\begin{quote}
    Is the title for Rule 6 missing some commas? `Communicate early, openly, and often enough'

\end{quote}

Agreed, we have added commas.


\begin{quote}
    I don’t think the `analysis' in `analysis' code' needs to be possessive in the actionable item of Rule 10.

\end{quote}

Agreed, we have updated this to simply read `the analysis code'.

\begin{quote}
    
Maybe a figure would be handy to summarize the 10 actionable items? It could be simply the 10 rule headings in boxes and e.g. Trello and GitHub logos beside Rule 5. This would make it easy for someone to show to collaborators at the start of a potential collaboration :-)

\end{quote}

We agree a figure summarizing the manuscript is needed, and potentially helpful during first meetings with potential collaborators. We note, however, that Figure 1 already is aimed to this, and contains all rule headers. Hence, we have not updated or added another figure.

\begin{quote}
    
The table is a bit confusing. The scale is like a 5 point Likert scale but there are two statements. Maybe some instruction at the top to put an `x' in the appropriate column, where 3 is `agree equally with both statements', 1 is `agree with statement on left strongly', and 5 is `agree with statement on right strongly'.

\end{quote}

Thank you, we have updated the table header to ease filling it out.\\

\noindent \textbf{Reviewer 3}

\begin{quote}
    Reviewer \#3: This article is a well-written and well-thought piece, written by authors who clearly have lots of experience with interdisciplinary research collaboration. I am excited to see it published in the future. I especially enjoyed and appreciated Table 1, and might use it in my own research practice. The quotes also nicely illustrated presented points.

My only suggestion is to consider adding a few personal anecdotes from the past collaborations, that would illustrate the rules and what happens when they are followed and/or broken.
\end{quote}

Thank you for the positive feedback. We considered adding anecdotes of past collaborations, but found them very difficult to anonymize and, particularly, to get the tone right. We found that the expectations form in Table~1 was a neutral alternative to this, as it was (sadly) inspired by past interactions. We also added the quotes to try to spice up the rule contents while keeping the desired tone. Hence, we find adding anecdotes a difficult task and would prefer to avoid it.


\begin{quote}
    
Minor comments:
1) "Without an aligned choice of collaborating labs, discrepancies in scientific interest and values are harder to navigate later on" => not clear what choice refers to here, please explain or elaborate
\end{quote}

Agreed, we have clarified the sentence.

\begin{quote}
    2) `What matters is excellent science - and hence high impact factor vs What matters is sound, robust science' => I think I understand the dichotomy the authors are presenting here, but I would suggest re-phrasing, as I can see two people with similar mindset choosing either option, especially in current publishing culture
\end{quote}

This item was meant to diagnose misalignments on what `excellent science' means, e.g. whether impactful (impact-factorwise) and surprising; or robust, reproducible and sound. In our experience, these two meanings are very polarized and a good indicator of unaligned scientific values. We have rephrased the sentence to clarify and tone it down.


\begin{quote}
    3) Is font size in Table 1 consistent?
\end{quote}

Yes it is, is the row (table cell) height which varies to make the table fit in one page.

\begin{quote}
    4) `To each his/her expertise' : could this be rephrased to be more inclusive, e.g. `their expertise'?
\end{quote}

We have updated the sentence.

\begin{quote}
5) Initially when reading the article, it was a bit confusing whether the advice was targeted from wet-lab to dry-lab scientists, or the other direction, and it felt like both directions were mentioned at times. Perhaps this could be unified.
\end{quote}

Thank you, we have updated the title and added a sentence in introduction to clarify this, also in line with Reviewer's 1 suggestion.

\end{document}
